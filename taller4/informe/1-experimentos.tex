\section{Métodos y condiciones de los experimentos}

Se desarrolló un script que a través de sucesivas consultas iterativas obtiene obtenga el registro MX de un dominio dado. Se consideró que cada consulta DNS puede tener 3 tipos de respuestas:

\begin{itemize}
\item Se devuelve los servidores DNS a los cuales seguir preguntando.
\item Se devuelve la respuesta a la consulta que se está haciendo.
\item Se devuelve el registro SOA de la zona indicando que el registro solicitado no forma parte de la base de datos de nombres de la zona.

\end{itemize}

Usando la herramienta desarrollada, se consultó por los servidores de mail que atienden los correos del dominio de cuatro universidades (sus nombres de dominio) en algún lugar del mundo. Además se analizó si los servidores de mail tienen nombres en el mismo dominio que el de la universidad o pertenecen a otro dominio. Además se intentó  averiguar también si dichos servidores de mail se encuentran en la misma zona geográfica. \#fixme: si se logra para las cuatro universidades, cambiar se "intentó" por se "averiguó".
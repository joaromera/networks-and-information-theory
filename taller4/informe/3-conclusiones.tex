\section{Conclusiones}


En la mitad de las consultas se necesitaron correr tres niveles de servidores DNS y en la otra mitad, cuatro, hasta obtener la información solicitada.
	
	¿Todos los servidores DNS Autoritativos que aparecen en las sucesivas respuestas responden a las consultas realizadas?
	
	Solo uno de los X servidores DNS Autoritativos consultados, nunca respondió. \#fixme: suponiendo que cada nivel consultado es un servidor DNS Autoritativo. Remplazar X con la cantidad de servidores DNS Autoritativos consultados.

	Para cuatro universidades, se encontraron nueve nombres de servidores de mail, de los cuales cuatro tienen nombres en el mismo dominio que la universidad. 

	Es interesante el caso de la Universidad de Tübingen, para la cual se encontraron cinco nombres de servidores de mail, de los cuales únicamente dos coinciden con el dominio de la universidad. Para el resto de las universidades se encontraron menos nombres de servidores de mail y todos ellos o coinciden con el dominio de la universidad o no coinciden. \#Fixme: no sé si queda claro el segundo elemento de este punto, lo que quiero decir es que el caso de la universidad alemana es la única que tiene algunos que coinciden y otros que no. Chequear que el nombre de la Universidad quede bien.
	\#fixme: se podría averiguar si los servidores son públicos o privados (supongo que todos los que pertenecen a las redes de interconexion universitarias son públicas y el resto no).


	La única universidad para la cual coinciden las IPs de los servidores de correo con la IPs del servidor Web, es para la Universidad Nacional Arturo Jauretche.

	Para las universidades argentinas, exceptuando el Departamento de Computación - FCEyN UBA, las IPs de los servidores de correo con la IPs del servidor Web, pertenecen a la Red de Interconexión Universitaria, en Argentina.

	Exceptuando la Universidad de Tübingen, las IPs de los servidores de correo y las de los servidores WEB, pertenecían a una mísma zona geográfica. Para la universidad alemana, los servidores se encuentran a 296 km.

	
	Se encontraron en total 15 direcciones IP distintas. \#Fixme: quizás aclarar a que corresponden (ej: 7 a servidores de mail 9 a web).
	\#fixme: nos falta ver esto:¿Estas direcciones IP corresponden a dispositivos que están prendidos? Usar ping para averiguarlo

	\#fixme: para las universidades argentias, se puede intentar responder si crearon la web antes o después de que exista la Red de Interconexión Universitaria. En 1997 se crea la RIU
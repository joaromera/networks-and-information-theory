\section{Conclusiones}

\begin{itemize}
  

  \item En la mitad de las consultas se necesitaron correr tres niveles de servidores DNS y en la otra mitad, cuatro, hasta obtener la información solicitada.
	
  \item Solo uno de todos los servidores DNS Autoritativos consultados nunca respondió. 

  \item Para cuatro universidades, se encontraron nueve nombres de servidores de mail, de los cuales cuatro tienen nombres en el mismo dominio que la universidad. 

  \item La única universidad para la cual coinciden las IPs de los servidores de correo con la IPs del servidor Web, es para la Universidad Nacional Arturo Jauretche.

  \item Para las universidades argentinas, exceptuando el Departamento de Computación - FCEyN UBA, las IPs de los servidores de correo con la IPs del servidor Web, pertenecen a la Red de Interconexión Universitaria, en Argentina.

  \item La Universidad de Tübingen, las IPs de los servidores de correo y las de los servidores WEB, pertenecían a una mísma zona geográfica. Para la universidad alemana, los servidores se encuentran a 296 km.

  \begin{itemize}
    \item Es interesante el caso de la Universidad de Tübingen para la
      cual se encontraron cinco nombres de servidores de mail, de los
      cuales únicamente dos coinciden con el dominio de la
      universidad. Esto puede deberse al uso de correos de respaldo,
      para tener robustez frente a un fallo de los servidores ubicados
      en la red de la universidad.
  \end{itemize}

  \item Se visitaron en total 14 servidores DNS distintos.

  % \item \#fixme: para las universidades argentias, se puede intentar responder si crearon la web antes o después de que exista la Red de Interconexión Universitaria. En 1997 se crea la RIU.

\end{itemize}

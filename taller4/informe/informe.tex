% ******************************************************** %
%              TEMPLATE DE INFORME ORGA2 v0.1              %
% ******************************************************** %
% ******************************************************** %
%                                                          %
% ALGUNOS PAQUETES REQUERIDOS (EN UBUNTU):                 %
% ========================================
%                                                          %
% texlive-latex-base                                       %
% texlive-latex-recommended                                %
% texlive-fonts-recommended                                %
% texlive-latex-extra?                                     %
% texlive-lang-spanish (en ubuntu 13.10)                   %
% ******************************************************** %
\documentclass[a4paper]{article}
\usepackage[spanish]{babel}
\usepackage[utf8]{inputenc}
\usepackage{charter}   % tipografia
\usepackage{graphicx}
%\usepackage{makeidx}
\usepackage{paralist} %itemize inline
\usepackage{algorithm}  % implementacion ondas en C
% sudo apt-get install texlive-science
\usepackage{algorithmic} % implementacion ondas en C
%\usepackage{float}
\usepackage{amsmath, amsthm, amssymb}
%\usepackage{amsfonts}
%\usepackage{sectsty}
%\usepackage{charter}
%\usepackage{wrapfig}
%\usepackage{listings}
%\lstset{language=C}
% \setcounter{secnumdepth}{2}
\usepackage{wrapfig}
\usepackage{underscore}
\usepackage{caratula}
\usepackage{url}
\usepackage{multicol}
\usepackage{caption}
\usepackage{hyperref}
\usepackage{tikz}


% ********************************************************* %
% ~~~~~~~~              Code snippets             ~~~~~~~~~ %
% ********************************************************* %
\usepackage{color} % para snipets de codigo coloreados
\usepackage{fancybox}  % para el sbox de los snipets de codigo
\definecolor{litegrey}{gray}{0.94}
\newenvironment{codesnippet}{%
	\begin{Sbox}\begin{minipage}{\textwidth}\sffamily\small}%
	{\end{minipage}\end{Sbox}%
		\begin{center}%
		\vspace{0cm}\colorbox{litegrey}{\TheSbox}\end{center}\vspace{0cm}}
% ********************************************************* %
% ~~~~~~~~         Formato de las páginas         ~~~~~~~~~ %
% ********************************************************* %
\usepackage{fancyhdr}
\pagestyle{fancy}
%\renewcommand{\chaptermark}[1]{\markboth{#1}{}}
\renewcommand{\sectionmark}[1]{\markright{\thesection\ - #1}}
\fancyhf{}
\fancyhead[LO]{Sección \rightmark} % \thesection\ 
\fancyfoot[LO]{\small{Centeno, Romera, Linari, Junqueras}}
\fancyfoot[RO]{\thepage}
\renewcommand{\headrulewidth}{0.5pt}
\renewcommand{\footrulewidth}{0.5pt}
\setlength{\hoffset}{-0.8in}
\setlength{\textwidth}{16cm}
%\setlength{\hoffset}{-1.1cm}
%\setlength{\textwidth}{16cm}
\setlength{\headsep}{0.5cm}
\setlength{\textheight}{25cm}
\setlength{\voffset}{-0.7in}
\setlength{\headwidth}{\textwidth}
\setlength{\headheight}{13.1pt}
\renewcommand{\baselinestretch}{1.1}  % line spacing
% ******************************************************** %

\begin{document}

\thispagestyle{empty}
\materia{Teoría de las Comunicaciones}
\submateria{Primer Cuatrimestre de 2020}
\titulo{Taller de Distributed Name System (DNS)}
\integrante{Facundo Linari}{591/16}{facundo.linari@hotmail.com}
\integrante{Joaquin P. Centeno}{699/16}{joaquinpcenteno@gmail.com}
\integrante{Joaquin Romera }{183/16}{joakromera@gmail.com}
\integrante{Juan Junqueras}{804/16}{junquerasjuan@gmail.com}

\maketitle

\newpage

\thispagestyle{empty}
\vfill
\thispagestyle{empty}
\vspace{3cm}

\tableofcontents
\newpage

\section{Introducción}

El objetivo del presente trabajo es entender los protocolos involucrados a nivel de red para la herramienta \textit{traceroute}. Para esto, se implementó una versión basada en los mensajes \textit{echo request/reply} del protocolo ICMP y se la utilizó para analizar rutas a sitios web de universidades en diferentes continentes.

Cada alumno utilizo la herramienta implementada con los siguientes sitios web:

\begin{itemize}
\item \url{www.tu-berlin.de} con host situado en Berlín (Alemania)
\item \url{www.waseda.jp} con host situado en Dallas (EEUU)
\item \url{www.exactas.uba.ar} con host situado en CABA (Argentina)
\item \url{www.uio.no} con host situado en Oslo (Noruega)
\end{itemize}
\newpage
\section{Métodos y condiciones de los experimentos}

fixme: máximo 400 palabras

fixme: Explicación del código.

Como se mencionó, el código utiliza la biblioteca \textit{Scapy} para capturar paquetes de la red (a la que está conectada la computadora en la que se corre el script) con el objetivo de caracterizarla como una fuente de información de memoria nula, desde ahora S1.

Capturamos paquetes usando la función sniff de Scapy, toma los mismos de la NIC default. Por cada paquete capturado se llama un callback que acumula estadísticas de los paquetes recibidos y los categoriza según dirección (broadcast, unicast) y protocolo. Para cada una de estas tuplas y con las estadísticas recolectadas al momento de cada callback se calcula la probabilidad y la información de cada una. Finalmente con esta información calculamos la entropía de la fuente.


El mismo capturará 30.000 paquetes de ethernet (fixme: esto es así según lo que leí de la documentación de haslayer), clasificándolos según su direccionamiento (\textit{BROADCAST} o \textit{UNICAST}) y el protocolo que utiliza protocolo. De esta manera un símbolo de esta fuente de información S1, estará determinado por la tupla \textit{direccionamiento, protocolo}. El script cuenta la cantidad de apariciones de cada símbolo, para posteriormente asignarle la probabilidad a cada uno (entendida como la inversa de la cantidad de apariciones del símbolo fixme: no se si queda claro). Posteriormente se calcula la información que aporta cada símbolo y la entropía de la fuente. 
fixme: describir el output del código.

fixme: Descripción de cada una de las 4 redes

Cada uno de los integrantes del grupo realizó corridas del script en los momentos que se describió en la introducción.

En la tabla \ref{fig:redesCaracteristicas} se detalla la red de cada uno de los integrantes. Todas estas estan conectadas via WI-FI.

\begin{center}
\begin{tabular}{| c | c |}
  \hline  
  Integrante & \# Hosts \\
  \hline			
  Centeno & 10 \\
  Romera & 7 \\
  Linari & 5 \\
  Junqueras & 3 \\
  \hline  
\end{tabular}
\captionof{figure}{Características de las redes utilizadas.}
\label{fig:redesCaracteristicas} 
\end{center}


\newpage
\section{Resultados de los experimentos}


La siguiente tabla reune los resultados los experimentos. (fixme: referencia a tabla)

TABLAS CON LOS RESULTADOS (si está en el informe borrar hasta “HASTAACÁ”)

Porcentaje de tráfico Broadcast/Unicast sobre el tráfico total (Fixme: pasar a tabla en latex)
15 hs
Red 1: Unicast 99.527%
Red 2: Unicast 100%
Red 3: Unicast 99.773%
Red 4: Unicast 92.967%

00hs
Red 1: Unicast 97.59%
Red 2: Unicast 99.976%
Red 3: Unicast 99.953%
Red 4: Unicast 99.54%

HASTAACÁ

Como puede verse en la misma (fixme: referencia a taba), en todas las redes hubo una fuerte preponderancia de tráfico Unicast respecto a Broadcast (fixme: agregar el porcentaje de Uni vs Broad, haciendo promedio entre todas las redes en todos los experimentos que pongamos en el informe). Este resultado se mantiene en todas las redes y en ambos momentos del día en los que ocurrieron los experimentos. Si bien esperabamos encontrar una disminución del porcentaje de tráfico Unicast durante la noche, lo mismo ocurrió solo en las redes 1 y 2, mientras que en las redes 3 y 4 subió. La única diferencia notable entre los experimentos diurnos y nocturnos fue la duración de los mismos: durante el día las capturas no tomaron más de 45 minutos (en algunos casos mucho menos), mientras que las nocturnas necesitaron varias horas para llegar a las 30.000 tramas capturadas.

Por otro lado, los resultados (fixme: referencia al gráfico o tabla que mejor lo muestre) muestran un fuerte predominio del protocolo 2048, este es el protocolo IPv4 que utiliza UDP y TCP, su función es transportar datos, por eso es esperable que sea el protocolo con más ocurrencias y cuyo símbolo tenga la probabilidad más alta en todo horario, especialmente en aquellos donde hay un uso más activo de la red. 
(Fixme: se puede explicar algo del resto de los protocoles encontrados).

La entropía de las redes analizadas aumenta cuanto más equiprobables sean los símbolos de la fuente, como se dijo y se puede ver en los resultados (fixme: hacer referencia a la tabla que corresponda) la probabilidad del protocolo 2048 es ampliamente superior al resto, el margen de disparidad disminuye (aunque menos de fixme: agregar propornción en ordenes de magnitud en la que cambia) respecto al resto de los protocolos cuando la red tiene un uso menos intensivo, por lo que la entropía disminuye, por lo que podría pensarse que hay una correlacón entre la intensidad del uso de la red (fixme: si tenemos los tiempos de todos los experimentos se puede medir la intensidad como la inversa del tiempo que tarda en conseguirse los 30000 paquetes) .

Cantidad de información de cada símbolo comparado con la entropía de la red (fixme: hacer alguna interpretación corta Y COMPARAR LA ENTROPÍA DE DIA VS NOCHE)
FIXME: ACA INCLUIR PLOT COMPARANDO ENTROPIA DIA VS NOCHE PARA CADA RED.

<fixme: meter un scatterplot X: cant hosts Y: Entropía y explicar algo de esta relación, en la conclsuión se menciona que la red 2 es la que menos entropía tiene, no estaría mal volver a mencionarlo acá>


La siguiente tabla muenstra la función de cada uno de los protocolos encontrados. fixmme: hacer referencia a tabla. En la misma se puede ver cuales son de control y cuáles transportan datos del usuario.

unicast 2048 Internet Protocol version 4 (IPv4)
unicast 34525   Internet Protocol version 6 (IPv6)
unicast 2054 Address Resolution Protocol (ARP)
unicast 34958 IEEE Std 802.1X - Port-based network access control
unicast 33024 Customer VLAN Tag Type (C-Tag, formerly called the Q-Tag) (initially Wellfleet)
Unicast 35020 Link Layer Discovery Protocol (LLDP)
Unicast 35130 IEEE 1905.1


broadcast 2048, 2054





Protocolos de control: 2054 ARP, 34958 IEEE Std 802.1X, 35020 Link Layer Discovery Protocol (LLDP), 33024 Customer VLAN Tag Type

Datos: 2048 IPv4 e 34525 IPv6















fixme: lo siguiente no lo borro porque creo que son tablas útiles (borrar cuando alguien sepa que no van).
****************************
Porcentaje de tráfico Broadcast/Unicast sobre el tráfico total.
Porcentaje de aparición de cada protocolo encontrado.
Entropía de cada red analizada.
Cantidad de información de cada símbolo comparado con la entropía de la red.


\subsection{Experimentos tomados durante las 15:00 Hs.}%
\label{sub:experimentos_1500}

A continuación, se expresan los resultados obtenidos en las mediciones
realizadas a las 15:00hs.

\begin{table}[h!]
    \centering
    \begin{tabular}{|c c c c|}
     \hline
    Destino & Protocolo & Probabilidad & Información \\
    \hline
    UNICAST   &  2048 & 0.96410 &  0.05275 \\
    UNICAST   & 34525 & 0.02617 &  5.25613 \\
    UNICAST   &  2054 & 0.00493 &  7.66322 \\
    BROADCAST &  2048 & 0.00350 &  8.15843 \\
    BROADCAST &  2054 & 0.00123 &  9.66322 \\
    UNICAST   & 34958 & 0.00007 & 13.87267 \\
    \hline
    \end{tabular}
    \caption{Red 1: Facundo Linari (15:00 Hs) Entropia: 0.26759}
    \label{table:1tarde}
\end{table}


\begin{table}[h!]
    \centering
    \begin{tabular}{|c c c c|}
     \hline
    Destino & Protocolo & Probabilidad & Información \\
    \hline
    UNICAST     &    2048       &   0.99980      &   0.00029          \\
    UNICAST     &      2054     &    0.00020      &   12.28771          \\
    \hline
    \end{tabular}
    \caption{Red 2: Juan Junqueras (15:00 Hs) Entropia: 0.00275}
    \label{table:2tarde}
\end{table}


\begin{table}[h!]
    \centering
    \begin{tabular}{|c c c c|}
     \hline
    Destino & Protocolo & Probabilidad & Información \\
    \hline
    unicast     &     2048      &     0.98883         &   0.01620          \\
    unicast     &     34525      &     0.00590          &     7.40507         \\
    broadcast     &    2054       &    0.00187          &   9.06532          \\
    unicast     &     35130      &     0.00140          &   9.48036          \\
    unicast      &     35020      &     0.00137          &   9.51512          \\
    broadcast      &  2048         &     0.00040         &   11.28771          \\
    unicast      &       33024    &    0.00013          &    12.87267         \\
    unicast      &     2054      &      0.00010        &     13.28771        \\
    \hline
    \end{tabular}
    \caption{Red 3: Joaquín Perez Centeno (15:00 Hs) Entropia: 0.11047}
    \label{table:3tarde}
\end{table}

\begin{table}[h!]
    \centering
    \begin{tabular}{|c c c c|}
     \hline
    Destino & Protocolo & Probabilidad & Información \\
    \hline
    UNICAST    &   2048        &     0.89153         &     0.16564        \\
    BROADCAST        &  2048         &   0.03960            &  4.65836           \\
    UNICAST       &    34525       &      0.03340         &   4.90401          \\
    BROADCAST       &  2054         &     0.03073          &    5.02405         \\
    UNICAST      &   2054        &     0.00470         &   7.73312          \\
    UNICAST     &    33024       &     0.00003          &    14.87267          \\
    \hline
    \end{tabular}
    \caption{Red 4: Joaquín Romera (15:00 Hs) Entropia: 0.68719}
    \label{table:4tarde}
\end{table}


\subsection{Experimentos tomados durante las 00:00 Hs.}%
\label{sub:experimentos_0000}

A continuación, se expresan los resultados obtenidos en las mediciones
realizadas a las 00:00hs.

\begin{table}[h!]
    \centering
    \begin{tabular}{|c c c c|}
     \hline
    Destino & Protocolo & Probabilidad & Información \\
    \hline
    UNICAST  &     2048      &    0.77777          &      0.36259     \\
    UNICAST     &    34525       &   0.18293          &  2.45061       \\
    BROADCAST    &       2054    &  0.01667        &     5.90689      \\
    UNICAST    &     2054      &    0.01433      &    6.12448        \\
    BROADCAST    &     2048      &   0.00743        &  7.07177     \\
    UNICAST      &     34958     &     0.00087       &   10.17224  \\
    \hline
    \end{tabular}
    \caption{Red 1: Facundo Linari (00:00 Hs) Entropia: 0.97792}
    \label{table:1noche}
\end{table}


\begin{table}[h!]
    \centering
    \begin{tabular}{|c c c c|}
     \hline
    Destino & Protocolo & Probabilidad & Información \\
    \hline
    UNICAST     &    2048       &   0.94147 & 0.08702   \\
    UNICAST     &      2054     &    0.05740 & 4.12281   \\
    UNICAST        &   34958        &   0.00060 & 10.70275     \\
    UNICAST        &      34525     &     0.00030 & 11.70275   \\
    BROADCAST        &   2054        &    0.00017 & 12.55075      \\
    BROADCAST        &     2048      &      0.00007 & 13.87267      \\
    \hline
    \end{tabular}
    \caption{Red 2: Juan Junqueras (00:00 Hs) Entropia: 0.33152}
    \label{table:2noche}
\end{table}


\begin{table}[h!]
    \centering
    \begin{tabular}{|c c c c|}
     \hline
    Destino & Protocolo & Probabilidad & Información \\
    \hline
    UNICAST     &     2048      &    0.99727        &   0.00395      \\
    UNICAST     &     34525      &     0.00143        &    9.44641     \\
    BROADCAST     &    2054       &  0.00047         &   11.06532      \\
    UNICAST     &     35130      &     0.00040          &  11.28771        \\
    UNICAST      &     35020      &     0.00040        &   11.28771          \\
    UNICAST      &  2054         &   0.00003       &   14.87267        \\
    \hline
    \end{tabular}
    \caption{Red 3: Joaquín Perez Centeno (00:00 Hs) Entropia: 0.03217}
    \label{table:3noche}
\end{table}


\begin{table}[h!]
    \centering
    \begin{tabular}{|c c c c|}
     \hline
    Destino & Protocolo & Probabilidad & Información \\
    \hline
    UNICAST    &   2048        &    0.98957 & 0.01513     \\
    UNICAST        &  34525         & 0.00463 & 7.75373       \\
    BROADCAST       &  2048         &    0.00393 & 7.99003      \\
    UNICAST      &   2054        &    0.00120 & 9.70275       \\
    BROADCAST     &    2054       &     0.00067 & 10.55075       \\
    \hline
    \end{tabular}
    \caption{Red 4: Joaquín Romera (00:00 Hs) Entropia: 0.10100}
    \label{table:4noche}
\end{table}

En la tabla \ref{table:desc_protocolos} se describen todos los protocolos encontrados en las redes analizadas, indicando si están destinados a control o transporte de datos.

\begin{table}[h!]
    %\centering
\begin{tabular}{| l | c | c | c |}
  \hline  
  Número & Protocolo & Transporte/Control & Función \\
  \hline			
  2048 & Internet Protocol version 4 & Transporte & Dar soporte  a la \\
       & (IPv4) &   &  capa de transporte       \\
       &        &   &                           \\
  \hline
  2054 & Address Resolution Protocol & Control  & Descubrir la dirección  \\
       & (ARP) &    &  de la capa de enlace asociada      \\
       &        &   &  a una dirección de red             \\
  \hline
  33024 & Shortest Path Bridging & Control & Es un spanning tree protocols \\
       & (IEEE 802.1aq) &    & (STP)        \\
       &        &   &                           \\
  \hline
  34525 & Internet Protocol Version 6 & Transporte & Misma que IPv4 \\
       & (IPv6) &   &         \\
       &        &   &                           \\
  \hline
  34958 & Extensible Authentication Protocol & Control & Proveer mecanismo de \\
       & over LAN &     &  autenticación a dispositivos        \\
       & (EAPOL - IEEE 802.1X)       &   & en la LAN o WLAN     \\
  \hline
  35020 & Link Layer Discovery Protocol & Control & Manejar y monitorear \\
       & (LLDP) &  & redes          \\
       &        &   &                           \\
  \hline
  35130 & IEEE 1905.1 & Control & Definir una capa de  \\
       &  &   &  abstracción para multiples  \\
       &        &   &  tecnologías de redes de hogar         \\
  \hline  
\end{tabular}
\caption{Descripción de cada protocolo visto en las redes analizadas.}
\label{table:desc_protocolos} 
\end{table}

\subsection{Proporcion \textsc{Unicast} y \textsc{Broadcast}}%
\label{sub:proporcion_unicast_y_broadcast}


\begin{table}[h!]
    \centering
    \begin{tabular}{|c c c c c|}
     \hline
         & Tarde     &         & Noche     &         \\ % re feo pero zafa.
     Red & Broadcast & Unicast & Broadcast & Unicast \\
    \hline
     flinari & 0.99527 & 0.00473 & 0.9759 & 0.0241 \\
     jjunqueras & 1.0 & 0.0 & 0.99977 & 0.00024 \\
     jcperez & 0.99773 & 0.00227 & 0.99953 & 0.00047 \\
     jromera & 0.92966 & 0.07033 & 0.9954 & 0.0046 \\
    \hline
    \end{tabular}
    %\caption{Red 1: Joaquín Romera (00:00 Hs) Entropia: 0.10100}
    %\label{table:1}
\end{table}

\newpage
\section{Conclusiones}

Consideramos que con la experimentación, conseguimos muestras que son una representación característica del comportamiento de la red en los horarios en los que fueron corridos los experimentos.
Es notable que durante los experimentos nocturnos, el tiempo necesario para juntar 30.000 frames fue mucho mayor.
Esto es consecuente con la diferencia de tráfico de las redes en estos horarios -habiendo un menor uso durante la noche-.
A pesar de la diferencia en el tiempo que requirieron los experimentos, no pudimos observar una tendencia común en la entropía alcanzada: dos de las redes disminuyeron la entropía por la noche, pero las dos restantes aumentaron.

De las cuatro redes estudiadas se destaca que la red 2, de menor cantidad de hosts, fue la que menor entropía registró en todos los experimentos. Si bien las muestras no son lo suficientemente grandes como para sacar conclusiones, podríamos hipotetizar que mientras menor sea la red en cantidad de hosts, menor sea la entropía que alcance.

En ningún de estos casos la entropía alcanzada se acercó a la máxima teórica. La diferencia en el horario para cada experimento tampoco arroja resultados consistentes, las redes 1 y 2 aumentaron la entropía alcanzada por la noche, mientras que la 3 y 4 la disminuyeron. La que más se acercó fue la red 1 durante la noche, superando apenas un tercio de la máxima teórica, luego la red 4 durante el día llegando a un cuarto de su respectiva teórica máxima.

Por otro lado, pudimos observar que en la mayoría de los casos la cantidad de tráfico Broadcast es insignificante respecto a Unicast, rara vez superando el 2\%. Únicamente en la red 4 el porcentaje de Broadcast se acerco al 10\%, pero se dispone de pocas muestras y la topología de la red es similar a las 1 y 2, con resultados diferentes. Por esto no podemos extraer más conclusiones que la preponderancia del tráfico Unicast en las redes hogareñas. Particularmente el protocolo que tuvo mayor incidencia en los datos fue el 2048 (IPv4), teniendo más del 90\% del total.

Finalmente consideramos que los experimentos realizados son suficientes para afirmar que en las topologías de redes hogareñas predomina el tráfico de paquetes Unicast, con un bajo nivel de entropía y que los protocolos de los mismos se encuentra reducido a un conjunto pequeño con dominio de los paquetes de datos de usuario.
+
\end{document}

\section{Resultados de los experimentos}

\subsection{Universidad Nacional Del Chaco Austral}

El host de la Universidad Nacional Del Chaco Austral es \textit{www.uncaus.edu.ar/}.

Los resultados son los siguientes:

\begin{table}[h]
    \centering
    \begin{tabular}{|l|l|l|l|l|}
        \hline
        \textbf{Timeout} & \textbf{UDP Cerrado} & \textbf{UDP Abierto | Filtrado} & \textbf{TCP Abierto} & \textbf{TCP Cerrado} \\ \hline
        \textbf{0.5}     & 455                  & 569                             & 3                    & 1014                 \\ \hline
        \textbf{1}       & 598                  & 426                             & 3                    & 1015                 \\ \hline
        \textbf{2.5}     & 785                  & 239                             & 3                    & 1013                 \\ \hline
        \textbf{5}       & 879                  & 145                             & 3                    & 1014                 \\ \hline
    \end{tabular}
\end{table}

fimxe: insertar resultados para cada una de la universidades y responder:
¿Cuántos puertos abiertos aparecen? ¿A que servicios/protocolos (nivel de aplicación) corresponden?
¿Cuántos puertos filtrados tenían los sitios web que se probaron?
¿Existen otros puertos bien conocidos que puedan estar abiertos en los hosts que se probaron?



\newpage

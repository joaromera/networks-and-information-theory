\section{Resultados de los experimentos}

\subsection{Universidad Nacional de Córdoba}%
\label{sub:universidad_nacional_de_cordoba}

\begin{table}[h]
    \centering
    \begin{tabular}{|llllll|}
        \hline
        \textbf{Timeout} & \textbf{UDP Cerrado} & \textbf{UDP Abierto | Filtrado} & \textbf{TCP Abierto} & \textbf{TCP Cerrado} & \textbf{TCP Filtrado} \\
        \hline
        \textbf{0.5}     & 369.0                & 655.0                           & 2.0                  & 1017.0               & 5.0                   \\
        \textbf{1.0}     & 533.0                & 491.0                           & 2.0                  & 1017.0               & 5.0                   \\
        \textbf{2.5}     & 742.0                & 282.0                           & 2.0                  & 1017.0               & 5.0                   \\
        \textbf{5.0}     & 855.0                & 169.0                           & 2.0                  & 1016.0               & 6.0                   \\
        \hline
    \end{tabular}
\end{table}

\newpage

\subsection{Universidad Nacional Del Chaco Austral}
\label{sub:universidad_nacional_del_chaco}

El host de la Universidad Nacional Del Chaco Austral es \textit{www.uncaus.edu.ar/}.

Los resultados son los siguientes:

\begin{table}[h]
    \centering
    \begin{tabular}{|llllll|}
        \hline
        \textbf{Timeout} & \textbf{UDP Cerrado} & \textbf{UDP Abierto | Filtrado} & \textbf{TCP Abierto} & \textbf{TCP Cerrado} & \textbf{TCP Filtrado} \\ \hline
        \textbf{0.5}     & 455                  & 569                             & 3                    & 1014                 & 7                     \\
        \textbf{1}       & 598                  & 426                             & 3                    & 1015                 & 6                     \\
        \textbf{2.5}     & 785                  & 239                             & 3                    & 1013                 & 8                     \\
        \textbf{5}       & 879                  & 145                             & 3                    & 1014                 & 7                     \\ \hline
    \end{tabular}
\end{table}

% ¿Cuántos puertos abiertos aparecen? ¿A que servicios/protocolos (nivel de aplicación) corresponden?

Lo primero que podemos notar es que para UDP la detección de puertos cerrados aumenta a medida que se incrementa el tiempo de timeout. Dar más tiempo de espera a una respuesta del puerto ayuda a disminuir los falsos positivos para el estado indeterminado de \textit{Abierto | Filtrado}.

\begin{figure}[h]
    \centering
    \includegraphics[width=0.75\textwidth]{./imagenes/uncaus.png}
    % \caption{}
    % \label{fig:timeoutUncaus}
\end{figure}


No se encontraron puertos UDP para los cuales se pudiese determinar con certeza que estuvieran abiertos. Si consideramos aquellos cuyo estado puede estar filtrado y los resultados de todas las pruebas con distintos timeouts podemos contabilizar 845 puertos con estado \textit{abierto o filtrado}. Sin embargo, observando los estados para el mayor tiempo de timeout, este número se reduce a 145.

En TCP se encontraron abiertos los puertos 22, 80 y 443 -independientemente del tiempo de timeout. En estos casos el estado registrado fue \textit{abierto} con \textit{Flag 18 (SA)}. Para el protocolo UDP el puerto 443 apareció como cerrado, mientras que los 22 y 80 no se pudo determinar si estaba abierto o filtrado. Estos son los puertos típicos para SSL, HTTP y HTTPS (en orden).

Para todos los otros puertos TCP se determinó que su estado era \textit{Cerrado} con \textit{Flag 20 (RA)}.

\newpage

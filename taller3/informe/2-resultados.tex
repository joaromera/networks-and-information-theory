\section{Resultados de los experimentos}

\subsection{Universidad Nacional Del Chaco Austral}

El host de la Universidad Nacional Del Chaco Austral es \textit{www.uncaus.edu.ar/}.

Los resultados son los siguientes:

\begin{table}[h]
    \centering
    \begin{tabular}{|l|l|l|l|l|}
        \hline
        \textbf{Timeout} & \textbf{UDP Cerrado} & \textbf{UDP Abierto | Filtrado} & \textbf{TCP Abierto} & \textbf{TCP Cerrado} \\ \hline
        \textbf{0.5}     & 455                  & 569                             & 3                    & 1014                 \\ \hline
        \textbf{1}       & 598                  & 426                             & 3                    & 1015                 \\ \hline
        \textbf{2.5}     & 785                  & 239                             & 3                    & 1013                 \\ \hline
        \textbf{5}       & 879                  & 145                             & 3                    & 1014                 \\ \hline
    \end{tabular}
\end{table}

% ¿Cuántos puertos abiertos aparecen? ¿A que servicios/protocolos (nivel de aplicación) corresponden?

Lo primero que podemos notar es que para UDP la detección de puertos cerrados aumenta a medida que se incrementa el tiempo de timeout. Dar más tiempo de espera a una respuesta del puerto ayuda a disminuir los falsos positivos para el estado indeterminado de \textit{Abierto | Filtrado}.

No se encontraron puertos UDP para los cuales se pudiese determinar con certeza que estuvieran abiertos. Si consideramos aquellos cuyo estado puede estar filtrado y los resultados de todas las pruebas con distintos podemos contabilizar 845 puertos con estado \textit{abierto o filtrado}. Sin embargo, observando los estados para el mayor tiempo de timeout, este número se reduce a 145.

En TCP se encontraron abiertos los puertos 22, 80 y 443 -independientemente del tiempo de timeout-. En estos casos el estado registrado fue \textit{abierto} con \textit{Flag 18 (SA)}. Para el protocolo UDP el puerto 443 apareció como cerrado, mientras que los 22 y 80 no se pudo determinar si estaba abierto o filtrado.

\newpage

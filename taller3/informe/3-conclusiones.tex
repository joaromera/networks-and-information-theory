\section{Conclusiones}

\begin{itemize}
    \item Para los casos en los que los puertos presentan un alto número de
        puertos filtrados, tales como la \textsc{FCEyN} y la \textsc{UNRN}, se
        hace evidente el uso de firewall.
    \item No se puede afirmar o descartar la presencia de un firewall en los
        casos de la \textsc{Universidad de Córdoba} y \textsc{Universidad
        Nacional del Chaco Austral}.
    \item Se dificulta identificar puertos \textsc{UDP} cerrados, pues, para
        \textit{timeouts} bajos se clasifican erróneamente como puertos
        abiertos o filtrados. Incluso para un timeout alto, no hay forma de
        saber si un puerto clasificado como \texttt{abierto|filtrado}
        corresponde a un puerto cerrado para el cual se perdió el paquete.
\end{itemize}

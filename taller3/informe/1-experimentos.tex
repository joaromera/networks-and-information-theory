\section{Métodos y condiciones de los experimentos}

\subsection{Explicación del código}

Como se mencionó, se utilizó la herramienta \textit{nmap}, con la mísma se implementó un script que dado el \textit{ip}  de un servidor, envía paquetes TCP y UDP sus primeros 1024 puertos, con el objetivo de identificar el estado de los mismos. 

Los estados posibles de los puertos para los paquetes TCP son \textit{abierto}, \textit{cerrado} y \textit{filtrado} mientras que para UDP a esos tres estados se le agrega el estado \textit{abierto|filtrado} (no se puede determinar cual de los dos es). El estado de un puerto se caracteriza por el mensaje (o falta del mismo) que el servidor envía al usuario. 

Por último cada paquete enviado tiene un \textit{timeout} que determina el tiempo que puede pasar un paquete en cada servidor, si este valor llega a cero, el paquete deja de transmitirse. Se experimetnó variando el valor del timeout con el objetivo de evaluar los cambios en las respuestas a los paquetes. 
\subsection{Características de las pruebas}

Cada integrante del grupo utilizo la herramienta implementada con los siguientes sitios web:

\begin{itemize}
  \item \url{www.unc.edu.ar}.
  \item \url{www.unrn.edu.ar}.
  \item \url{www.unrn.edu.ar}.
  \item \url{www.exactas.uba.ar}  
\end{itemize}

Los resultados de las corridas se detallan en la sección resultados.
\section{Conclusiones}

Consideramos que con la experimentación, conseguimos muestras que son una representación característica del comportamiento de la red en los horarios en los que fueron corridos los experimentos.
Es notable que durante los experimentos nocturnos, el tiempo necesario para juntar 30.000 frames fue mucho mayor.
Esto es consecuente con la diferencia de tráfico de las redes en estos horarios -habiendo un menor uso durante la noche-.
A pesar de la diferencia en el tiempo que requirieron los experimentos, no pudimos observar una tendencia común en la entropía alcanzada: dos de las redes disminuyeron la entropía por la noche, pero las dos restantes aumentaron.

De las cuatro redes estudiadas se destaca que la red 2, de menor cantidad de hosts, fue la que menor entropía registró en todos los experimentos. Si bien las muestras no son lo suficientemente grandes como para sacar conclusiones, podríamos hipotetizar que mientras menor sea la red en cantidad de hosts, menor sea la entropía que alcance.

En ningún de estos casos la entropía alcanzada se acercó a la máxima teórica. La diferencia en el horario para cada experimento tampoco arroja resultados consistentes, las redes 1 y 2 aumentaron la entropía alcanzada por la noche, mientras que la 3 y 4 la disminuyeron. La que más se acercó fue la red 1 durante la noche, superando apenas un tercio de la máxima teórica, luego la red 4 durante el día llegando a un cuarto de su respectiva teórica máxima.

Por otro lado, pudimos observar que en la mayoría de los casos la cantidad de tráfico Broadcast es insignificante respecto a Unicast, rara vez superando el 2\%. Únicamente en la red 4 el porcentaje de Broadcast se acerco al 10\%, pero se dispone de pocas muestras y la topología de la red es similar a las 1 y 2, con resultados diferentes. Por esto no podemos extraer más conclusiones que la preponderancia del tráfico Unicast en las redes hogareñas. Particularmente el protocolo que tuvo mayor incidencia en los datos fue el 2048 (IPv4), teniendo más del 90\% del total.

Finalmente consideramos que los experimentos realizados son suficientes para afirmar que en las topologías de redes hogareñas predomina el tráfico de paquetes Unicast, con un bajo nivel de entropía y que los protocolos de los mismos se encuentra reducido a un conjunto pequeño con dominio de los paquetes de datos de usuario.
+
\section{Conclusiones}

Consideramos que con la experimentación, conseguimos muestras que son una feaciente representación del comportamiento de la red en los horarios en los que fueron corridos los experimentos.
Es notable que durante los experimentos nocturnos, el tiempo necesario para juntar 30.000 frames fue mucho mayor, debido al menor uso de la red. Para capturas más lentas, se observó una mayor entropía, dado que se capturaron más paquetes de los protocolos menos frecuentes.

(fixme: Incluir un parrafo caracterizando las redes en general y las características de las mísmas conn su entropía.)
Tambien es interesante destacar que la red 2, es la que tiene menos hosts y la que menor entropía presenta, quizás no es una muestra suficientemente grande como para sacar conclusiones muy fuertes, pero lleva a pensar que cuanto menor sea la red, menor será la entropía.

Por otro lado, con este trabajo podemos observar que la cantidad de tráfico Broadcast es 
Insignificante respecto a  Unicast (fixme: ver que en todo el informe escribamos broadcast y uni con mayúscula al principio), el único caso en el que la probailidad de los paquetes de broadcast superó el (fixme: agregar cota superior de broadcast inmediatamente menor a 20\% -es decir el otro record de broadcast más alto sin contar este en el que alcanzó el 20\%) es en el que habia un solo host en la red en un horario de nula actividad el, allí el porocentaje de trafico broadcast llego al 20\%. (fixme: se puede agregar una interpretación de esto).

Fixme: estos bullets quizás se pueden agregar (pero no en forma de bullet, sino con desarrollo) a la conclusión:
-en todas las muestras ganó 2048. IPv4.
-en horarios de mayor actividad siempre estuvo arriba del 90\% de los paquetes totales
-según lo que estuvimos viendo 2048 se envía cuando el usuario usa intenet, el resto se mandan esporádicamente
-eso es otro aspecto a mencionar, durante la noche hay  menor actividad y los experimentos llevaron considerablemnte más tiempo
-también me parece que mientras más tiempo duró el experimento mayor es la entropía. (parece ser inversamete proporconal al tiempo fixme: usar esto si tenemos lo datos y los gráficos que lo muestren tiempo)
-mientras más tiempo tardamos en llegar a esas tramas aparece una proporción mayor de los otros protocolos -> Esto es porque en un tiempo largo, aparecen mas cosas de menor probabilidad.

En ningún caso la entropía alcanzada se acercó a la máxima teórica. La diferencia en el horario para cada experimento tampoco arroja resultados consistentes, las redes 1 y 2 aumentaron la entropía alcanzada por la noche, mientras que la 3 y 4 la disminuyeron.

La que más se acercó fue la red 1 durante la noche, superando apenas un tercio de la máxima teórica, luego la red 4 durante el día llegando a un cuarto de su respectiva teórica máxima. (fixme: estaría bueno poner estas tablas en resultados y en conclusiones simplemente hacer referencia a las mismas).

fixme: acá va la tabla que está en el goolge docs, debajo de "estaTabla":


Un resultado interesante fue la aparición de algunos paquetes inesperados, como por ejemplo Unicast 35130 en la red 3 (exclusivamente), el 33024, que sólo apareció en las redes 3 y red 4.

Según lo que pudimos encontrar, el paquete Unicast 35130 (fixme: intentar explicar que hace y si no lo encontramos, hipotetizar sobre por que no lo encontramos: solo encontré data en http://standards-oui.ieee.org/ethertype/eth.txt. Ni lo encuentro como un posible valor en el código de scapy (https://github.com/secdev/scapy/blob/master/scapy/libs/ethertypes.py en hexa 893A). ). Por otro lado, el paquete 33024 (fixme: explicar paquete https://en.wikipedia.org/wiki/IEEE_802.1Q)

(Fixme: ver si se responde a esta pregunta en algún lugar del informe: ¿En alguna red la entropía de la fuente alcanza la entropía máxima teórica?)



%\section{Bibliografía}
\section{Resultados de los experimentos}


En las subsecciones \ref{sub:experimentos_1500} y \ref{sub:experimentos_0000} se reunen los resultados de los experimentos de la tarde y de la noche respectivamente.

\subsection{Experimentos tomados durante las 15:00 Hs.}
\label{sub:experimentos_1500}

A continuación, se expresan los resultados obtenidos en las mediciones
realizadas a las 15:00hs:

\begin{table}[h!]
    \centering
    \begin{tabular}{|c c c c|}
     \hline
    Destino & Protocolo & Probabilidad & Información \\
    \hline
    UNICAST   &  2048 & 0.96410 &  0.05275 \\
    UNICAST   & 34525 & 0.02617 &  5.25613 \\
    UNICAST   &  2054 & 0.00493 &  7.66322 \\
    BROADCAST &  2048 & 0.00350 &  8.15843 \\
    BROADCAST &  2054 & 0.00123 &  9.66322 \\
    UNICAST   & 34958 & 0.00007 & 13.87267 \\
    \hline
    \end{tabular}
    \captionof{table}{Red 1: Facundo Linari (15:00 Hs) Entropia: 0.26759}
    \label{table:1tarde}
\end{table}


\begin{table}[h!]
    \centering
    \begin{tabular}{|c c c c|}
     \hline
    Destino & Protocolo & Probabilidad & Información \\
    \hline
    UNICAST     &    2048       &   0.99980      &   0.00029          \\
    UNICAST     &      2054     &    0.00020      &   12.28771          \\
    \hline
    \end{tabular}
    \captionof{table}{Red 2: Juan Junqueras (15:00 Hs) Entropia: 0.00275}
    \label{table:2tarde}
\end{table}


\begin{table}[h!]
    \centering
    \begin{tabular}{|c c c c|}
     \hline
    Destino & Protocolo & Probabilidad & Información \\
    \hline
    UNICAST     &     2048      &     0.98883         &   0.01620          \\
    UNICAST     &     34525      &     0.00590          &     7.40507         \\
    BROADCAST     &    2054       &    0.00187          &   9.06532          \\
    UNICAST     &     35130      &     0.00140          &   9.48036          \\
    UNICAST      &     35020      &     0.00137          &   9.51512          \\
    BROADCAST      &  2048         &     0.00040         &   11.28771          \\
    UNICAST      &       33024    &    0.00013          &    12.87267         \\
    UNICAST      &     2054      &      0.00010        &     13.28771        \\
    \hline
    \end{tabular}
    \captionof{table}{Red 3: Joaquín Perez Centeno (15:00 Hs) Entropia: 0.11047}
    \label{table:3tarde}
\end{table}

\begin{table}[h!]
    \centering
    \begin{tabular}{|c c c c|}
     \hline
    Destino & Protocolo & Probabilidad & Información \\
    \hline
    UNICAST    &   2048        &     0.89153         &     0.16564        \\
    BROADCAST        &  2048         &   0.03960            &  4.65836           \\
    UNICAST       &    34525       &      0.03340         &   4.90401          \\
    BROADCAST       &  2054         &     0.03073          &    5.02405         \\
    UNICAST      &   2054        &     0.00470         &   7.73312          \\
    UNICAST     &    33024       &     0.00003          &    14.87267          \\
    \hline
    \end{tabular}
    \captionof{table}{Red 4: Joaquín Romera (15:00 Hs) Entropia: 0.68719}
    \label{table:4tarde}
\end{table}


\subsection{Experimentos tomados durante las 00:00 Hs.}%
\label{sub:experimentos_0000}

A continuación, se expresan los resultados obtenidos en las mediciones
realizadas a las 00:00hs:

\begin{table}[h!]
    \centering
    \begin{tabular}{|c c c c|}
     \hline
    Destino & Protocolo & Probabilidad & Información \\
    \hline
    UNICAST  &     2048      &    0.77777          &      0.36259     \\
    UNICAST     &    34525       &   0.18293          &  2.45061       \\
    BROADCAST    &       2054    &  0.01667        &     5.90689      \\
    UNICAST    &     2054      &    0.01433      &    6.12448        \\
    BROADCAST    &     2048      &   0.00743        &  7.07177     \\
    UNICAST      &     34958     &     0.00087       &   10.17224  \\
    \hline
    \end{tabular}
    \captionof{table}{Red 1: Facundo Linari (00:00 Hs) Entropia: 0.97792}
    \label{table:1noche}
\end{table}


\begin{table}[h!]
    \centering
    \begin{tabular}{|c c c c|}
     \hline
    Destino & Protocolo & Probabilidad & Información \\
    \hline
    UNICAST     &    2048       &   0.94147 & 0.08702   \\
    UNICAST     &      2054     &    0.05740 & 4.12281   \\
    UNICAST        &   34958        &   0.00060 & 10.70275     \\
    UNICAST        &      34525     &     0.00030 & 11.70275   \\
    BROADCAST        &   2054        &    0.00017 & 12.55075      \\
    BROADCAST        &     2048      &      0.00007 & 13.87267      \\
    \hline
    \end{tabular}
    \captionof{table}{Red 2: Juan Junqueras (00:00 Hs) Entropia: 0.33152}
    \label{table:2noche}
\end{table}


\begin{table}[h!]
    \centering
    \begin{tabular}{|c c c c|}
     \hline
    Destino & Protocolo & Probabilidad & Información \\
    \hline
    UNICAST     &     2048      &    0.99727        &   0.00395      \\
    UNICAST     &     34525      &     0.00143        &    9.44641     \\
    BROADCAST     &    2054       &  0.00047         &   11.06532      \\
    UNICAST     &     35130      &     0.00040          &  11.28771        \\
    UNICAST      &     35020      &     0.00040        &   11.28771          \\
    UNICAST      &  2054         &   0.00003       &   14.87267        \\
    \hline
    \end{tabular}
    \captionof{table}{Red 3: Joaquín Perez Centeno (00:00 Hs) Entropia: 0.03217}
    \label{table:3noche}
\end{table}


\begin{table}[h!]
    \centering
    \begin{tabular}{|c c c c|}
     \hline
    Destino & Protocolo & Probabilidad & Información \\
    \hline
    UNICAST    &   2048        &    0.98957 & 0.01513     \\
    UNICAST        &  34525         & 0.00463 & 7.75373       \\
    BROADCAST       &  2048         &    0.00393 & 7.99003      \\
    UNICAST      &   2054        &    0.00120 & 9.70275       \\
    BROADCAST     &    2054       &     0.00067 & 10.55075       \\
    \hline
    \end{tabular}
    \captionof{table}{Red 4: Joaquín Romera (00:00 Hs) Entropia: 0.10100}
    \label{table:4noche}
\end{table}


\subsection{Proporcion \textsc{Unicast} y \textsc{Broadcast}}
\label{sub:proporcion_unicast_y_broadcast}

En el cuadro \ref{table:proporcion_unicast_y_broadcast} se muestra el porcentaje de tráfico \textit{Broadcast}/\textit{Unicast} sobre el tráfico total.

\begin{table}[h!]
    \centering
    \begin{tabular}{|c|c|c|}
     \hline
         & Tarde   & Noche   \\ % re feo pero zafa.
     Red & Unicast & Unicast  \\
    \hline
     1 & 99.52\% &  97.59\%  \\
     2 & 100\%  & 99.97\%  \\
     3 & 99.77\%  &  99.95\%  \\
     4 & 92.96\%  & 99.54\%  \\
    \hline
     Promedio & 98.06\%  & 99.26\%  \\
    \hline
    \end{tabular}
    \captionof{table}{Porcentaje de cada tipo de destinatario en cada experimento.}
    \label{table:proporcion_unicast_y_broadcast}
\end{table}

Como puede verse en el cuadro \ref{table:proporcion_unicast_y_broadcast}, en
todas las redes hubo una fuerte preponderancia de tráfico \textit{Unicast}
respecto a \textit{Broadcast}. Este resultado se mantiene en todas las redes y
en ambos momentos del día en los que ocurrieron los experimentos. Si bien se
esperaba encontrar una disminución del porcentaje de tráfico \textit{Unicast}
durante la noche, lo mismo ocurrió solo en las redes 1 y 2, mientras que en las
redes 3 y 4 subió. La única diferencia notable entre los experimentos diurnos y
nocturnos fue la duración de los mismos: durante el día las capturas no tomaron
más de 45 minutos (en algunos casos mucho menos), mientras que las nocturnas
necesitaron varias horas para llegar a las 30.000 tramas capturadas.

Por otro lado, los resultados expresados en las secciones
\ref{sub:experimentos_1500} y \ref{sub:experimentos_0000} muestran un fuerte
predominio del protocolo 2048, este es el protocolo \textsc{IPv4} para el cual
era esperable que sea el protocolo con más ocurrencias y cuyo símbolo tenga la
probabilidad más alta en todo horario, especialmente en aquellos donde hay un
uso más activo de la red.

La entropía de las redes analizadas aumenta cuanto más equiprobables sean los
símbolos de la fuente, como se comentó y puede ver en los resultados de las
secciones \ref{sub:experimentos_1500} y \ref{sub:experimentos_0000}, la
probabilidad del protocolo 2048 es superior al resto. La disparidad disminuye
respecto al resto de los protocolos cuando la red tiene un uso menos intensivo,
por lo que la entropía aumenta. Se podría pensar que existe una correlación
entre la intensidad del uso de la red y la entropía calculada en los
experimentos.

\subsection{Descripción de protocolos}

En el cuadro \ref{table:desc_protocolos} se muestran las funciones de los protocolos encontrados. En la misma se puede ver cuales son de control y cuáles transportan datos del usuario.

\begin{table}[h!]
    %\centering
\begin{tabular}{| l | c | c | c |}
  \hline  
  Número & Protocolo & Transporte/Control & Función \\
  \hline			
  2048 & Internet Protocol version 4 & Transporte & Dar soporte  a la \\
       & (IPv4) &   &  capa de transporte       \\
       &        &   &                           \\
  \hline
  2054 & Address Resolution Protocol & Control  & Descubrir la dirección  \\
       & (ARP) &    &  de la capa de enlace asociada      \\
       &        &   &  a una dirección de red             \\
  \hline
  33024 & Shortest Path Bridging & Control & Es un spanning tree protocols \\
       & (IEEE 802.1aq) &    & (STP)        \\
       &        &   &                           \\
  \hline
  34525 & Internet Protocol Version 6 & Transporte & Misma que IPv4 \\
       & (IPv6) &   &         \\
       &        &   &                           \\
  \hline
  34958 & Extensible Authentication Protocol & Control & Proveer mecanismo de \\
       & over LAN &     &  autenticación a dispositivos        \\
       & (EAPOL - IEEE 802.1X)       &   & en la LAN o WLAN     \\
  \hline
  35020 & Link Layer Discovery Protocol & Control & Manejar y monitorear \\
       & (LLDP) &  & redes          \\
       &        &   &                           \\
  \hline
  35130 & IEEE 1905.1 & Control & Definir una capa de  \\
       &  &   &  abstracción para multiples  \\
       &        &   &  tecnologías de redes de hogar         \\
  \hline  
\end{tabular}
\captionof{table}{Descripción de cada protocolo visto en las redes analizadas.}
\label{table:desc_protocolos} 
\end{table}

Un resultado interesante fue la aparición de algunos paquetes inesperados, como por ejemplo \textit{Unicast} 35130 en la red 3 (exclusivamente), y el 33024, que sólo apareció en las redes 3 y red 4.
El paquete \textit{Unicast} 35130 define una capa de abstracción común para tecnologías hogareñas de origen heterogeneo descriptas en los siguientes protocolos IEEE 1901, IEEE 802.11, IEEE 802.3 and MoCA 1.1.
El 33024 es un protocolo para administración de Virtual Lans en redes Ethernet IEEE 802.3.

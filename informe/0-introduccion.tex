\section{Introducción}

El objetivo del presente trabajo es utilizar técnicas y conceptos provistos por la teoría de la información para estudiar los diversos protocolos de la red de manera empírica. En el mismo, se analizarán cada una de las redes hogareñas con tecnología WiFi (802.11) de los miembros del grupo de trabajo, buscando modelarlas como fuentes de información de memoria nula. Para esto se utilizó un script que utiliza la biblioteca \textsc{Scapy}, un sniffer que permite capturar y analizar los paquetes que se transmiten en la red. Luego de presentar los resultados de estos experimentos concluiremos con algunas reflexiones sobre los mismos.

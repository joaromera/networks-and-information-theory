\section{Resultados de los experimentos}

\subsection{TU Berlin}

El host de la Technische Universität Berlin es www.tu-berlin.de


\begin{codesnippet}
    \begin{verbatim}
TTL IP              RTT             RTT salto       Outlier o negativo
  1 192.168.0.1           12.524138       12.524138          0
  6 181.96.120.69         37.382096       24.857958          0
  8 181.96.120.63         38.295614        0.913518          0
  9 190.216.88.33         58.547193       20.251579          0
 10 67.17.94.249         169.194187      110.646994          1
 12 4.69.142.209         283.095530      113.901343          1
 13 195.122.181.62       312.227178       29.131648          0
 14 188.1.144.58         362.458020       50.230842          1
 15 188.1.235.118        365.694586        3.236566          0
 16 130.149.126.78       332.631753      -33.062833          1
 17 130.149.7.201        354.637073       22.005320          0
 18 130.149.7.201        697.860956      343.223883          1
 19 130.149.7.201        387.589455     -310.271502          1
\end{verbatim}
\captionof{figure}{Ejemplo de sálida del script implementado. Notar que los mensajes que no son respondidos no se imprimen.}
\label{fig:output}
\end{codesnippet}

Dentro de Argentina las ips llegaron hasta la 190.216.88.33. En este tramo los tiempos de delay se mantienen cercanos entre sí.
El primer salto lo vemos con la ip 67.17.94.249. Esta ip se ubica dentro de Estados Unidos, las próximas ips también están en EEUU: 4.69.142.209 y 195.122.181.62.
Los tiempos suben pero se encuentran en el rango de los 300-400 ms una vez que sucede el salto interoceánico hacia Europa, a partir de la ip 188.1.144.58.
Respondieron 13 de los 19 nodos 68,42\%. Según el método de Cimbala los cables submarinos podrían haber sido 67.17.94.249 (entre Argentina y EEUU) y el 188.1.144.58 (entre Estados Unidos y Alemania).

\subsection{Universidad de Ciencia y Tecnología de Oman}

Con host en www.nu.edu.om, la Universidad de Ciencia y Tecnología de Oman se encuentra en Muscat, Oman, en la península arábiga, al occidente de Asia. A continuación estos fueron los tiempos registrados durante el traceroute:

\begin{codesnippet}
    \begin{verbatim}
TTL IP              RTT             RTT salto       Outlier o negativo
  1 192.168.0.1           12.560573       12.560573          0
  6 181.96.120.69         32.290440       19.729867          0
  8 181.89.2.125          32.738047        0.447607          0
  9 190.216.88.33         59.194783       26.456737          0
 10 67.17.94.249         163.833876      104.639092          0
 12 4.69.210.62          293.512807      129.678931          0
 13 195.122.181.190      319.292054       25.779247          0
 15 62.75.3.174          412.566829       93.274775          0
 16 82.178.159.6         488.297505       75.730677          0
 17 5.37.58.213          484.294157       -4.003348          1
 18 134.0.200.70         462.750559      -21.543598          1

Cantidad de saltos RTT positivos: 9

Cantidad de outliers: 0
\end{verbatim}
\captionof{figure}{Ejemplo de sálida del script implementado. Notar que los mensajes que no son respondidos no se imprimen.}
\label{fig:output}
\end{codesnippet}

Esta ruta presenta los mismos saltos iniciales hasta EEUU que la ruta previa. Llama la atención el desvío hacia California para volver a Dallas antes de hacer el salto hacia Europa, donde recae en Grecia para llegar a Oman. Otro dato curioso es que el método de Cimbala no detectó outliers.

\subsection{Norway - Oslo - www.uio.no - 129.240.118.130}

Los resultados del traceroute de la Universidad de Oslo, en Noruega, son los siguientes:

\begin{codesnippet}
    \begin{verbatim}
TTL IP              RTT             RTT salto       Outlier o negativo
1 192.168.1.1           43.158112       43.158112          1
2 181.88.172.15         60.790395       17.632283          0
3 181.88.169.84         52.491941       -8.298453          1
5 181.96.120.63         58.968106        6.476165          0
6 190.216.88.33         55.229664       -3.738442          1
7 67.17.99.233         297.179484      241.949821          1
9 4.69.207.33          346.348180       49.168695          1
10 213.248.84.80        335.634136      -10.714044          1
11 62.115.119.230       363.302803       27.668667          0
12 80.91.248.157        371.916800        8.613997          0
13 213.248.85.174       455.756232       83.839433          1
14 109.105.97.140       435.823846      -19.932386          1
15 109.105.97.126       456.400024       20.576178          0
16 109.105.102.67       491.082668       34.682644          0
17 128.39.230.18        478.868084      -12.214584          1
18 128.39.65.18         500.868043       21.999959          0
19 129.240.100.69       528.428482       27.560438          0
20 129.240.100.66       484.626974      -43.801507          1
21 129.240.25.162       491.957708        7.330734          0
22 129.240.118.130      477.704773      -14.252935          1
23 129.240.118.130      492.330074       14.625301          0

Cantidad de saltos RTT positivos: 14

Cantidad de outliers: 4
\end{verbatim}
\captionof{figure}{Ejemplo de sálida del script implementado. Notar que los mensajes que no son respondidos no se imprimen.}
\label{fig:output}
\end{codesnippet}

El primer gran salto en los tiempos de respuesta RTT se da en Louisiana, EEUU con ip 67.17.99.233. Cimbala lo marca como un outlier y efectivamente debe pertenecer a un salto interoceánico. Luego hay par de saltos dentro de EEUU: Virginia, California, antes de llegar a Inglaterra.
Este último salto, con destino 213.248.85.174, lleva un tiempo de 455ms, más alto que todos los anteriores, al mismo tiempo el método Cimbala también lo marca como un outlier.


\subsection{Universidad de Buenos Aires, FCEyN}

Por último nos dirigimos hacia la Facultad de Ciencias Exactas y Naturales, de la Universidad de Buenos Aires. Estos son las métricas obtenidas:

\begin{codesnippet}
  \begin{verbatim}
TTL IP              RTT             RTT salto       Outlier o negativo
  1 192.168.0.1          123.123407      123.123407          1
  6 181.96.120.69         97.155519      -25.967889          1
  8 181.96.120.63         87.048044      -10.107474          1
  9 200.49.69.157         80.490503       -6.557541          1
 10 131.100.186.97        81.974669        1.484165          0
 14 157.92.47.53          93.981446       12.006778          0
 16 157.92.32.18          82.148285      -11.833161          1

Cantidad de saltos RTT positivos: 3

Cantidad de outliers: 1
\end{verbatim}
\captionof{figure}{Ejemplo de sálida del script implementado. Notar que los mensajes que no son respondidos no se imprimen.}
\label{fig:output}
\end{codesnippet}

Tras localizar las direcciones IP de los nodos que ubicamos, llama la atención la ubicación del correspondiente a la dirección 131.100.186.97, en Paraguay. Los tiempo se mantienen relativamente estables en todos los nodos, incluso para el salto fuera de Argentina. No obtuvimos outliers y efectivamente toda la ruta fue dentro del continente.
\section{Resultados de los experimentos}

\subsection{TU Berlin}

El host de la Technische Universität Berlin es www.tu-berlin.de


\begin{codesnippet}
    \begin{verbatim}
TTL IP              RTT             RTT salto       Outlier o negativo
  1 192.168.0.1           12.524138       12.524138          0
  6 181.96.120.69         37.382096       24.857958          0
  8 181.96.120.63         38.295614        0.913518          0
  9 190.216.88.33         58.547193       20.251579          0
 10 67.17.94.249         169.194187      110.646994          1
 12 4.69.142.209         283.095530      113.901343          1
 13 195.122.181.62       312.227178       29.131648          0
 14 188.1.144.58         362.458020       50.230842          1
 15 188.1.235.118        365.694586        3.236566          0
 16 130.149.126.78       332.631753      -33.062833          1
 17 130.149.7.201        354.637073       22.005320          0
 18 130.149.7.201        697.860956      343.223883          1
 19 130.149.7.201        387.589455     -310.271502          1
\end{verbatim}
\captionof{figure}{Ejemplo de sálida del script implementado. Notar que los mensajes que no son respondidos no se imprimen.}
\label{fig:output}
\end{codesnippet}

Dentro de Argentina las ips llegaron hasta la 190.216.88.33. En este tramo los tiempos de delay se mantienen cercanos entre sí.
El primer salto lo vemos con la ip 67.17.94.249. Esta ip se ubica dentro de Estados Unidos, las próximas ips también están en EEUU: 4.69.142.209 y 195.122.181.62.
Los tiempos suben pero se encuentran en el rango de los 300-400 ms una vez que sucede el salto interoceánico hacia Europa, a partir de la ip 188.1.144.58.
Respondieron 13 de los 19 nodos 68,42\%. Según el método de Cimbala los cables submarinos podrían haber sido 67.17.94.249 (entre Argentina y EEUU) y el 188.1.144.58 (entre Estados Unidos y Alemania).
\section{Resultados de los experimentos}

\subsection{TU Berlin}

El host de la Technische Universität Berlin es www.tu-berlin.de


\begin{codesnippet}
    \begin{verbatim}
TTL IP              RTT             RTT salto       Outlier o negativo
  1 192.168.0.1           12.524138       12.524138          0
  6 181.96.120.69         37.382096       24.857958          0
  8 181.96.120.63         38.295614        0.913518          0
  9 190.216.88.33         58.547193       20.251579          0
 10 67.17.94.249         169.194187      110.646994          1
 12 4.69.142.209         283.095530      113.901343          1
 13 195.122.181.62       312.227178       29.131648          0
 14 188.1.144.58         362.458020       50.230842          1
 15 188.1.235.118        365.694586        3.236566          0
 16 130.149.126.78       332.631753      -33.062833          1
 17 130.149.7.201        354.637073       22.005320          0
 18 130.149.7.201        697.860956      343.223883          1
 19 130.149.7.201        387.589455     -310.271502          1
\end{verbatim}
\captionof{figure}{Ejemplo de sálida del script implementado. Notar que los mensajes que no son respondidos no se imprimen.}
\label{fig:output}
\end{codesnippet}

Dentro de Argentina las ips llegaron hasta la 190.216.88.33. En este tramo los tiempos de delay se mantienen cercanos entre sí.
El primer salto lo vemos con la ip 67.17.94.249. Esta ip se ubica dentro de Estados Unidos, las próximas ips también están en EEUU: 4.69.142.209 y 195.122.181.62.
Los tiempos suben pero se encuentran en el rango de los 300-400 ms una vez que sucede el salto interoceánico hacia Europa, a partir de la ip 188.1.144.58.
Respondieron 13 de los 19 nodos 68,42\%. Según el método de Cimbala los cables submarinos podrían haber sido 67.17.94.249 (entre Argentina y EEUU) y el 188.1.144.58 (entre Estados Unidos y Alemania).

\subsection{Universidad de Ciencia y Tecnología de Oman}

Con host en www.nu.edu.om, la Universidad de Ciencia y Tecnología de Oman se encuentra en Muscat, Oman, en la península arábiga, al occidente de Asia. A continuación estos fueron los tiempos registrados durante el traceroute:

\begin{codesnippet}
    \begin{verbatim}
TTL IP              RTT             RTT salto       Outlier o negativo
  1 192.168.0.1           12.560573       12.560573          0
  6 181.96.120.69         32.290440       19.729867          0
  8 181.89.2.125          32.738047        0.447607          0
  9 190.216.88.33         59.194783       26.456737          0
 10 67.17.94.249         163.833876      104.639092          0
 12 4.69.210.62          293.512807      129.678931          0
 13 195.122.181.190      319.292054       25.779247          0
 15 62.75.3.174          412.566829       93.274775          0
 16 82.178.159.6         488.297505       75.730677          0
 17 5.37.58.213          484.294157       -4.003348          1
 18 134.0.200.70         462.750559      -21.543598          1

Cantidad de saltos RTT positivos: 9

Cantidad de outliers: 0
\end{verbatim}
\captionof{figure}{Ejemplo de sálida del script implementado. Notar que los mensajes que no son respondidos no se imprimen.}
\label{fig:output}
\end{codesnippet}

Esta ruta presenta los mismos saltos iniciales hasta EEUU que la ruta previa. Llama la atención el desvío hacia California para volver a Dallas antes de hacer el salto hacia Europa, donde recae en Grecia para llegar a Oman. Otro dato curioso es que el método de Cimbala no detectó outliers.

\subsection{Norway - Oslo - www.uio.no - 129.240.118.130}


\begin{codesnippet}
    \begin{verbatim}
TTL IP              RTT             RTT salto       Outlier o negativo
  1 129.240.118.130      249.198258      249.198258          1
  2 129.240.118.130      230.263948      -18.934309          1
  4 129.240.118.130      195.032835      -35.231113          1
  6 181.96.120.69        111.418798      -83.614037          1
  7 181.96.120.69        478.892326      367.473528          1
  8 181.89.2.125         116.725239     -362.167087          1
 10 67.17.99.233         249.275798      132.550559          1
 12 4.69.207.29          198.870659      -50.405139          1
 13 213.248.84.80        250.132317       51.261658          0
 14 62.115.120.176       250.017172       -0.115145          1
 15 62.115.143.121       244.544927       -5.472244          1
 16 213.248.85.174       410.669770      166.124843          1
 17 109.105.97.142       443.390125       32.720355          0
 18 109.105.97.65        433.815522       -9.574603          1
 19 109.105.97.50        428.418777       -5.396745          1
 20 109.105.97.56        423.805060       -4.613717          1
 21 109.105.97.133       414.118917       -9.686142          1
 22 109.105.102.67       445.133203       31.014285          0
 23 128.39.230.18        451.262739        6.129536          0
 24 128.39.65.18         451.991706        0.728967          0
 25 129.240.100.69       438.489095      -13.502612          1
 26 129.240.100.66       434.527073       -3.962022          1
 27 129.240.25.162       434.232510       -0.294563          1
 28 129.240.118.130      453.266022       19.033511          0
 29 129.240.118.130      365.932286      -87.333736          1

Cantidad de saltos RTT positivos: 10

Cantidad de outliers: 4
\end{verbatim}
\captionof{figure}{Ejemplo de sálida del script implementado. Notar que los mensajes que no son respondidos no se imprimen.}
\label{fig:output}
\end{codesnippet}
\section{Conclusiones}


% No-detección de saltos transatlanticos
Solo fueron detectados saltos transatlanticos para las universidades de Berlin
y Oslo. No fueron detectados en el caso de Oman y Buenos Aires. Se obtuvieron
falsos negativos, donde hubieramos esperado encontrar saltos transatlánticos.
% Falsos positivos:
A su vez, hubo varios falsos positivos, posiblemente por tratarse de nodos de
mucho tráfico, alta congestión.
% Conclusión:
\textit{Tracerouting} y el método de Cimbala no son infalibles para encontrar
estos saltos, hace falta un análisis supervisado.


% Nodos que no contestan ICMP
En todos los experimentos fueron detectados nodos que no responden paquetes con
\textsc{ICMP}. Seguramente es porque están configurados para ignorar estos
mensajes, por motivos de \textit{performance} o seguridad. La metodología
empleada no sirve para obtener información acerca de ellos.

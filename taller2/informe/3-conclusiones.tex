\section{Conclusiones}



% Todas las rutas pasan por Estados Unidos
Es interesante ver que todas las rutas que tienen como destino una ip de Europa o Medio Oriente, pasan por Estados Unidos, incluso aunque existan otras formas de llegar a estos destinos (como puede verse en cualquier mapa de conexiones transatlánticas). Esto nos hace sospechar que las conexiones más rápidas se encuentran en Estados Unidos.

% Hipótesis sobre las bifurcaciones de la ruta
También llamativo ver que en rutas como las de los casos 3.1 y 3.2, parece haber una bifurcación en el camino, pero una de las opciones es abandonada. Hipotetizamos que se debe a que el algorítmo está averiguando que nodos conocen la ip de destino.

% Ruta más rápida no necesariamente es la más corta.
En el caso 3.3 la ruta hace un desvío hacia el oeste, para luego continuar hacia el este, que no parece tener sentido en términos de distancia, por esto nos lleva hipotetizar distancias como la del ancho de Estados Unidos, son insignificantes a la hora de encontrar la ruta más rápida. Es decir, la ruta más corta (en distancia), no siempre es la más corta.

% No-detección de saltos transatlanticos
Solo fueron detectados saltos transatlanticos para las universidades de Berlin
y Oslo. No fueron detectados en el caso de Oman y Buenos Aires. Se obtuvieron
falsos negativos, donde hubieramos esperado encontrar saltos transatlánticos.

% Falsos positivos:
A su vez, hubo varios falsos positivos, posiblemente por tratarse de nodos de
mucho tráfico, alta congestión.
% Conclusión:
\textit{Tracerouting} y el método de Cimbala no son infalibles para encontrar
estos saltos, hace falta un análisis supervisado.


% Nodos que no contestan ICMP
En todos los experimentos fueron detectados nodos que no responden paquetes con
\textsc{ICMP}. Seguramente es porque están configurados para ignorar estos
mensajes, por motivos de \textit{performance} o seguridad. La metodología
empleada no sirve para obtener información acerca de ellos.


% Repeticiones de ips
Puede repetirse el destinatario en las ultimas \textsc{TLL} de varios
experimentos. Esto se debe a que para distintas iteraciones se obtuvieron
caminos de distinta duración. Se ve que la longitud y los nodos intermedios de
la ruta son dinámicos.

\section{Conclusiones}

Notas

Saltos transatlánticos detectados: las universidades de Berlin y Oslo.
No se detectaron en el caso de Oman y Buenos Aires.
Hubo varios falsos positivos, posiblemente por tratarse de nodos de mucho tráfico, alta congestión.
También falsos negativos, como en el caso de la universidad de Oman, donde hubieramos esperado encontrar saltos transatlánticos.
Tracerouting y el método de Cimbala no son 100\% confiables para encontrar estos saltos, hace falta un análisis supervisado.

En todos los casos hubo nodos que no respondieron los echo-reply o los time-exceeded. Seguramente es porque están configurados para ignorar estos mensajes, por motivos de performance o seguridad.